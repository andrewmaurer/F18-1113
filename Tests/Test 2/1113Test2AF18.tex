% ${framed}

\documentclass[9pt,addpoints]{exam}
\usepackage{dsfont}
\usepackage{amsfonts}
\usepackage{amsmath}
\usepackage{array}
\usepackage{tabularx}
\usepackage{etoolbox}
\usepackage{eso-pic}
\usepackage{hyperref}
\usepackage{color}

\usepackage[letterpaper,top=1.55cm, bottom=3.5cm, outer=4cm, inner=3cm,
            heightrounded, marginparwidth=0cm,marginparsep=1cm]{geometry}

\newtoggle{compress}
\toggletrue{compress}
%\togglefalse{compress}

\hypersetup{
  pdftitle={Test 2 for Math 1113, Fall 2018},
  pdfsubject={Precalculus},
  pdfauthor={Andrew Maurer},
  pdfkeywords={},
  anchorcolor = {red},
  colorlinks = {true},
  %% pdfpagemode={FullScreen}
}
 

\usepackage{tikz}
\usepackage{pgf}
\usepackage{pgfplots}
\usepgfplotslibrary{fillbetween}
%%\pgfplotsset{width=7cm}
\usepgfplotslibrary{patchplots}


%%\oddsidemargin=-0.5in
%%\evensidemargin=-0.5in
%%\textwidth=7.5in

%%\topmargin=-0.75in
%%\textheight=9.5in

\pagestyle{plain}
 \reversemarginpar

\pointpoints{pt}{pts}
\bracketedpoints

\def\Course{Math 1113}
\def\NumSec{}
%% \def\time{12:00pm-12:50pm}
\def\Date{Fall 2018}
\def\test{Test 2}
\def\version{}


 \pagestyle{headandfoot}
 \headrule
 \firstpageheader{\bf 
   University of Georgia \\ % 
   Department of Mathematics}
 {\bf\Course \\ \test  }
 {\Date}
  \runningheadrule
  \runningheader{\Course}
               {\test %%, Section \NumSec
               }
               {\Date}
 \footrule
 \firstpagefooter{\version}
                 {Page \thepage\ of \numpages}
                 {}
 \runningfooter  {\version}
                 {Page \thepage\ of \numpages}
                 {\rule{0cm}{.35cm} Points earned: \hbox to
                 1cm{\hrulefill} \\  %
                  out of a possible \pointsonpage{\thepage} points}
%%%%%%%%%%%%%%%%%%%%%%%%%%%%%%%%%%%%%%%%%%%%%%%%%%%%%%%%%%%%%%%%%%%
 \extrawidth     { 1.00in}
  \extraheadheight{0.15in}%[-0.30in]{-0.30in}
 \extrafootheight{-0.41in}
 \addpoints
 %%%%%%%%%%%%%%%%%%%%%%%%%%%%%%%%%%%%%%%%%%%%%%%%%%%%%%%%%%%%%%%%%%%

\newcommand{\pointBox}{\marginpar[\begin{minipage}{1cm}\hfill\vspace*{1cm}
    ~ \\ \rule{1.5cm}{0.2mm}\end{minipage}]{~}}


\newcommand{\laplace}[1]{\makebox{$ {\cal L} \{ #1 \}$}}
\newcommand{\vi}{\makebox{$\vec{\imath}$}}
\newcommand{\vj}{\makebox{$\vec{\jmath}$}}
\newcommand{\vk}{\makebox{$\vec{k}$}}
\newcommand{\lp}{\left(}
\newcommand{\rp}{\right)}
%%\newcommand{\half}{\frac{1}{2}}




\begin{document}


%% <%include file="coverPage.template"/>

\noindent
\begin{tabular}{ll@{\hspace{3em}}ll@{\hspace{3em}}ll}
  %%%%\multicolumn{4}{l}{\textit{You must acknowledge that this is your
  %%%%work by signing:} } \\ %

  \multicolumn{4}{m{\textwidth}}{
  By providing my signature below  I acknowledge
  that I  abide by the University's academic honesty policy.
  This is my work, and I did not get any help from anyone else
  during the exam:} \\ [25pt] \\
  
  Name (sign):    & \rule{5cm}{0.2mm} & 
                                        Name (print):   & \rule{5cm}{0.2mm}  \\ [20pt]
  Student Number: & \rule{5cm}{0.2mm} \\ [20pt]
  Instructor's Name: & \rule{5cm}{0.2mm} & Class Time: & \rule{4cm}{0.2mm}
\end{tabular}


%%\bonustotalformat{Common Knowledge}

\vqword{
  \begin{minipage}[h]{5em}
    ~ \\Problem \\ Number \\ [-8pt]
  \end{minipage}
}
\vpword{
  \begin{minipage}[h]{5em}
    ~ \\ Points \\ Possible \\ [-8pt]
  \end{minipage}
}
%%\cvbpword{
%%  \begin{minipage}[h]{5em}
%%    ~ \\ Bonus \\ Points \\ [-8pt]
%%  \end{minipage}
%%}
\vsword{  \begin{minipage}[h]{5em}
    ~ \\ Points \\ Made \\ [-8pt]
  \end{minipage}
}

\newenvironment{nagging}%
  {\begin{itemize}%
    \setlength{\itemsep}{6pt}%
    \setlength{\parskip}{0pt}}%
  {\end{itemize}}

\vfill

\begin{tabular}{p{8cm}m{8cm}}
  \gradetable[v][questions]
  %%\partialgradetable{main}[v][questions]
  %%\partialbonusgradetable{bonus}
  %%\combinedgradetable[v][questions]
  &
    \begin{nagging}
    \item If you need extra space use the last page.

    \item Please show your work. \textbf{An unjustified answer may receive
        little or no credit.}

    \item If you make use of a theorem to justify a conclusion then
      state the theorem used by name.

    \item Your work must be {\bf neat}. If I can't read it (or can't
      find it), I can't grade it.

    \item The total number of possible points that is assigned for
      each problem is shown here.  The number of points for each
      subproblem is shown within the exam.

    \item Please turn off your mobile phone.

    %\item You are only allowed to use a TI-30 calculator. No other
    %  calculators are permitted.

    \item A calculator is not necessary, but numerical answers should
      be given in a form that can be directly entered into a
      calculator.



    \end{nagging}
\end{tabular}

\vfill



\bonuspointpoints{Point Common Knowledge}{Bonus}


\clearpage 

%% <%include file="defaults.template"/>
%% <%include file="overallreport.template"/>

\bonuspointpoints{Point Common Knowledge}{Points Bonus}

\clearpage


\begin{questions}

   \question Use the function below to answer the following:
   
   \[f(x) = \dfrac{13 - 2x}{22+5x}\]
\vskip 0.1in
 
    \begin{parts}
    \part[7]  Determine $f^{-1}(x)$.
    \iftoggle{compress}{\vfill}{~\\}
    
    \part[7]  Determine the range of $f(x)$.  Give your answer in interval notation.

    \iftoggle{compress}{\vfill}{~\\}

  \end{parts}


  \iftoggle{compress}{\clearpage}

 \question Use the function below to answer questions 2(a), 2(b), and 2(c).  
 
 \[f(x) = \dfrac{\log_5(x+8) - 1}{x^2-8}\]
 \vskip 0.1in
 
  \begin{parts}
    \part[5]  Determine the \textbf{$y$-intercept(s)} of $f(x)$.  Give a decimal value correct to 4 places.

    %% <%include file="questionResults.template"/>    

    \iftoggle{compress}{\vfill}{~\\}

    \part[6]  Determine the \textbf{$x$-intercept(s)} of $f(x)$.  

    %% <%include file="questionResults.template"/>    

    \iftoggle{compress}{\vfill}{~\\}

    \part[10]  Determine the \textbf{domain} of $f(x)$.  Give your answer in interval notation.

    %% <%include file="questionResults.template"/>    

    \iftoggle{compress}{\vfill}{~\\}

  \end{parts}


  \iftoggle{compress}{\clearpage}

 \question[7] Let $f(x) = -12 + \dfrac{3}{5}(4)^{3 - 2x}$. Determine the value of $f^{-1}(2)$.

\vskip 4in

 \question[7] Suppose that $\log_c(p) = 3.4$ and that $\log_c(q) = 2.6$.  Determine the numeric value for the expression below:
 
\[ \log_c\left(\dfrac{c^2 \hskip 0.05in q^4}{p^7}\right)\]

\vskip 4in



  \iftoggle{compress}{\clearpage}
  

   \question Solve for $x$ for the following equations, or explain why an answer does not exist.

  \begin{parts}
    \part[7]  $e^{2x} -2e^{x} - 35 = 0$

    %% <%include file="questionResults.template"/>    

    \iftoggle{compress}{\vfill}{~\\}

    \part[7]  $\ln(3x) - \ln(2x + 3) = 2$

    \iftoggle{compress}{\vfill}{~\\}

  \end{parts}


  \iftoggle{compress}{\clearpage}
  
\question[11] A bank account compounds continuously at a rate of 3.2\%.  Assuming no deposits or withdrawals are made, determine the initial balance needed in order for the account to grow to \$52,000 in 30 years.  Give an exact answer or an answer correct to the nearest cent.

\vskip 4in 
  
  
  
\question[11] A bank account compounds twice a month at an annual rate of 2.2\%.  Assuming no deposits or withdrawals are made, determine how long it will take for the account balance to triple.  Give an exact answer, or an answer correct to four decimal places.

  \iftoggle{compress}{\clearpage}


\question[15] The amount of a radioactive compound in a sample decays exponentially.  The sample initially contains 110g of the compound, and after three years contains 83g.  How long will it take until there is only 55g of material remaining?  Give an exact answer, or an answer correct to two decimal places.


  
  \iftoggle{compress}{\clearpage}
    

\end{questions}



Extra space for work. \textbf{Do not detach this page.} If you want us to consider the work on this
    page you should print your name, instructor and class meeting time below. \\ [10pt]
    Name (print): \rule{3cm}{0.2mm} Instructor (print):
    \rule{3cm}{0.2mm}  Time: \rule{3cm}{0.2mm}


    \iftoggle{compress}{\vfill}{~\\}



\end{document}


