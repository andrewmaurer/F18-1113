\documentclass[letterpaper,11pt]{article}
\usepackage[margin=0.5in]{geometry}
\usepackage{hyperref}
\usepackage{ifpdf}
\usepackage{graphicx}
\usepackage{array}
\usepackage{multicol}

% \usepackage{draftwatermark}
% \SetWatermarkScale{4.0}

\pagestyle{empty}

\renewcommand{\baselinestretch}{1.0}

\begin{document}
\noindent\textbf{Precalculus (Math 1113).} \hfill Andrew Maurer
\\
\url{www.math.uga.edu/1113} \hfill \href{mailto:abm41450@uga.edu}{\texttt{abm41450@uga.edu}}
\\
\href{http://andrewmaurer.github.io/teaching}{\texttt{andrewmaurer.github.io/teaching}} \hfill Boyd 434H
\\

\noindent Welcome to the Fall 2018 section of Math 1113,
Precalculus. The course is designed to offer a broad introduction
to the topics necessary to succeed in calculus. We will examine a
range of issues from the definition of function, exponential and
logarithmic functions, and trigonometric functions. The goal is not to
solve particular equations. Our goal is to understand the different
techniques and approaches.


% \vspace{1em} \noindent
% We will explore the following topics:\vspace{.5em}\\
% \begin{tabular}{|l@{\hspace{1em}}p{30em}|} \hline
%   \textit{Topic} & \textit{Important Ideas} \\  \hline \hline
%   Function                & Determine the relationship between dependent
%                             and independent variables. Determine the
%                             range and domain of a given function. \\ \hline
%   Inverse Function        & Determine an inverse function and relate
%                             it to the original function. \\ \hline
%   Exponential Functions   & Define functions that model various
%                             phenomena and compare to other relationships
%                             such as linear and quadratic functions. \\ \hline
%   Logarithmic Functions   & Relate logarithmic functions to exponential 
%                             functions and solve equations with both
%                             exponential and logarithmic terms. \\ \hline
%   Trigonometric Functions & Relate trigonometric functions to the unit
%                             circle, define functions that model
%                             physical phenomena, solve equations with
%                             trigonometric terms, and define inverse
%                             functions for trigonometric functions. \\ \hline
% \end{tabular}

% \vspace{1em} \noindent
% Our evaluation is based on the following expectations:\vspace{.5em} \\
% \begin{tabular}{|l@{\hspace{2em}}p{31em}|} \hline
%   \textit{Quality of Work}    & \textit{Expectations} \\ \hline\hline
%   Needs Improvement  & Cannot identify basic equations 
%                        Cannot determine solutions for basic systems of
%                      equations \\ \hline
%   Satisfactory       & Can identify and solve all basic equations 
%                        Can determine solutions of all basic equations \\ \hline
%   Good               & Derive own systems \\
%                      & Determine solutions and stability of own systems \\ \hline
%   Excellent          & Tie together different concepts to
%                        solution techniques 
%                        Can determine solution to any one system using
%                        a variety of techniques \\ \hline
% \end{tabular}


%Here is the basic information for this course:


\begin{multicols}{3}
  \begin{center}
  \noindent \emph{Course No. 36903}\\
  T/R 9:30am -- 10:45am\\
  Forestry 4, 516 \columnbreak

  \noindent \emph{Course No. 15230}\\
  T/R 11:00am -- 12:15pm\\
  Life Sciences, B118\columnbreak

  \noindent \emph{Office Hours:}\\
  Tuesday 12:30pm -- 1:30pm\\
  Wednesday 10:00am -- 12:00pm\columnbreak
  \end{center}
\end{multicols}

\begin{description}
  
\item[Textbook:] \emph{Precalculus}, by Julie Miller and Donna Gerken, McGraw
  Hill (ISBN: 978-1-30-700456-4). A special edition for UGA is available at a reduced rate. You will need access to the \emph{ALEKS 360}
  homework system which is included with the UGA edition of the
  book. I will send out a financial aid code for two weeks of \emph{ALEKS} access while you decide whether or not to take the course.

\item[Description:] Preparation for calculus, including an intensive
  study of algebraic, exponential, logarithmic, and trigonometric
  functions and their graphs. Applications include simple
  maximum/minimum problems, exponential growth and decay, and
  surveying problems.

  A central idea is the definition of functions including the ability
  to work the range and domain of a function as well as the
  inverse. You should be able to work with exponential and logarithmic
  functions, be able to solve equations with exponents, and know the
  relationship between the exponential and logarithmic functions. You
  should be able to work with the definitions of the trigonometric
  functions, the unit circle, and be able to work with the inverses of
  the basic trigonometric functions.

\item[Course Goals:] Be able to define functions that describe various
  physical phenomena. Be able to manipulate relationships to isolate
  particular quantities of interest. Demonstrate a working knowledge
  of the domain and range of a function and the relationship between
  the range and domain.

% \item[Tutoring Resources:] There are a number of resources available to
%   you to help you succeed in the course. These include your instructor's
%   office hours, mathematics department study hall, and free tutoring
%   through the division of academic enhancement.
  
\item[Attendance:] Students who have more than three unexcused absences
  will be withdrawn from the course with a grade of W before the
  midpoint of the term. After the midpoint for the term the grade will
  be an F. If you repeatedly leave class early or
  arrive late it may be counted as an absence. 

\item[Announcements:] You are responsible for all announcements made in
  class regardless of your attendance.

\item[Correspondence:] If you would like to contact me, please use my
  UGA email address \href{mailto:abm41450@uga.edu}{\texttt{abm41450@uga.edu}} \emph{from your UGA email
  address}. I do my best to return all emails quickly. If
  I have not responded to your email within 48 hours, please assume I
  did not receive the email and try sending it again.

\item[Homework:] Homework will be assigned throughout the course, some on \emph{ALEKS} and some on paper. You will have an account set up on \emph{ALEKS}. You will find a link to \emph{ALEKS}
from the course ELC web page. If you have a problem with
the website please make use of the help resources at \emph{ALEKS}. \emph{I will email you a financial aid code for two weeks of ALEKS access while you decide whether to continue with the course.}

\item[Grading:] %~ \\ %\samepage
  
  The final grades are calculated using the following distribution:
\begin{multicols}{2}
  \begin{center}
  \begin{tabular}{c  c}
    3 In-Class Tests & 45\% \\
    Homework & 15\% \\
    Quizzes and Activities  & 10\% \\
    Basic Skills Tests & 5\% \\
    Final Exam & 25\%
  \end{tabular}
  \end{center}
  \columnbreak
  \begin{center}
    \begin{tabular}{|l r | l r|}
      \hline
      A & 92\% & C+ & 77\% \\
      A-- & 89\%&  C & 72\% \\
      B+ & 87\% & C-- & 69\% \\
      B & 82\% & D & 60\% \\
      B-- & 79\% & F & $<$ 60\%\\
      \hline
    \end{tabular}
  \end{center}
\end{multicols}

  If your final exam is higher than your lowest exam score from the
  first three exams, then the lowest exam score will be replaced with
  %the average of the lowest exam score and 
  the final exam score. This
  is only an option for students who maintain good standing in the
  course and maintain regular attendance, at the instructor's discretion.

\item[Test Dates:] Tests will take place at normal class time and are tentatively scheduled as follows:
  \begin{center}
    ~ \hfill September 13 \hfill October 11 \hfill November 15 \hfill ~
  \end{center}

  %The final exam will take place on DATE 3 from TIME 1 to TIME 2.

\item[Basic Skills Tests:] In addition to written tests there will be
  basic skills tests that will take place in the Mathematics
  Department's testing center. These will be tests on \emph{ALEKS}, and
  the focus is on calculations and basic ideas. There will be four
  rounds of tests. In each round there will be two tests, and your
  grade for each round will be the higher of the two grades. The dates for
  these tests will be announced in class.

\item[Calculator Policy:] The recommended calculator for the course is
  the TI-30xs. It is available at the book store, many retail outlets,
  and many on-line sites. You may not use a calculator that can
  perform any basic algebra steps. \textit{You may use any calculator in class but
  may only use approved calculators on quizzes or tests.}

\item[Make up Policy:] The right to miss a scheduled exam and take a
  make up exam can be awarded only by your instructor, and will be
  awarded rarely and only for a serious cause.  \emph{Do not count
    on being able to make up a test until you have explicit permission
    from your instructor. }  If for some reason you must miss an exam,
  you must apply in writing \emph{before} the exam.

\item[Late Submission:] If you submit work after a deadline without obtaining
  permission then you will not receive any credit for the assignment. 

\item[Grading Issues:]  Questions about grading of any work should be submitted
  to your instructor within one week of the return of the
  work.



%\item[Other Important Numbers]
%    ~ \\
%    \begin{tabular}{l}
%      Department of Mathematics and Computer Science      \\
%      357 Science Center    \\
%      268-2395
%    \end{tabular}

% \chardef\home=126

% \item[Web-Quizzes:] There will be seven web based quizzes available on
%   Webassign. The lowest web based quiz will be dropped in the
%   calculation of your final score.  The quizzes will consist of ten
%   problems and will be similar to the book or Webassign problems. The
%   due dates will be made available on Webassign. There will also be in
%   class quizzes which will generally be announced before they are
%   given. There may be additional unannounced quizzes.

\item[Quizzes:] In-class quizzes can be announced or unannounced. A student who is late for a quiz will not receive additional time.

\item[Electronics:] Use of cell phones and other electronic devices is limited to relevant academic use. If a student misuses an electronic device, they will be asked to  put it away. Repeated or consistent use of electronic devices for irrelevant or nonacademic reasons, or use that subtracts from the learning atmosphere, will result in the student's dismissal from the class meeting.

\item[Academic Accommodations:] If you require any kind of special
  accommodation please see your instructor.  Requests for academic
  accommodations should be made as soon as possible and at least one
  week prior to a graded activity to insure that we provide the proper
  resources.  Students must register with the Disability Resource
  Center, to verify their eligibility for appropriate accommodations.
  
\item[Academic Integrity:] As a University of Georgia student, you have
  agreed to abide by the University's academic honesty policy, ``A
  Culture of Honesty,'' and the Student Honor Code. All academic work
  must meet the standards described in ``A Culture of Honesty'' found
  at:
  \url{https://ovpi.uga.edu/academic-honesty/academic-honesty-policy}. Lack
  of knowledge of the academic honesty policy is not a reasonable
  explanation for a violation. Questions related to course assignments
  and the academic honesty policy should be directed to the
  instructor.




%\item[Topics covered] ~
%


\end{description}
\vfill
\begin{center}
\small{\emph{The course syllabus is a general plan for the course;
  deviations announced to the class by the instructor may be
  necessary.}}
\end{center}
\end{document}
