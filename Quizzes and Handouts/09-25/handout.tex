\documentclass{article}

\usepackage{enumerate}
\usepackage{amsmath,amsthm,amssymb}
\usepackage{tikz}
\usepackage{pgfplots}
\usepackage{multicol}
%\pgfplotsset{compat=newest}

\usepackage[margin=0.5in]{geometry}

\begin{document}

\noindent \textbf{Name:}\underline{\hspace{2in}} \hfill \textbf{Handout September 25}
\\ \\

\begin{enumerate}
\item (Multiple choice) A point $P(a,b)$ is on the graph of a function $f$ means that:
\begin{multicols}{2}
  \begin{enumerate}
    \setlength\itemsep{.5em}
  \item $a = f(b)$
  \item $f(b) - f(a) = 0$
  \item $b = f(a)$
  \item $\text{AROC} = (f(b) - f(a)) / (b-a)$
  \end{enumerate}
\end{multicols}
\item Find the inverse functions to the following functions, or explain why no such function exists. (Be sure to specify the domain of the inverse function.)
  \begin{enumerate}
    \setlength\itemsep{1in}
  \item \[\alpha(x) = \frac{10-x}{10 + x}\]
  \item ~
    \begin{table}[h]
      \centering
      \begin{tabular}{|c|c c c c c c|}
        \hline
        $x$ & ``please'' & ``excuse'' & ``my'' & ``dear'' & ``aunt'' & ``Sally'' \\
        \hline
        $q(x)$ & ``()'' & ``$\wedge$'' &  $\times$ & $\div$ & $+$ & $-$ \\
        \hline
      \end{tabular}
    \end{table}
  \item
    \[
      \text{ReLU}(x) =
      \begin{cases}
        x &\text{ if } x \geq 0 \\
        0 &\text{ if } x < 0
      \end{cases}
    \]
    \vspace{1in}
    
  \end{enumerate}

\item If we know $j$ is a one-to-one function, with $j(1) = 1$, $j(2) = 4$, $j(5) = 25$, and $j(6) = 30$. List four points that \emph{must} lie on the graph of $j^{-1}$.
  \pagebreak
\item You have some money, and will be depositing it into the bank for several years at an interest rate of $r$. Put the following compounding methods in order from least-interest-gained to most-interest-gained.
  \begin{multicols}{2}
  \begin{enumerate}
  \item Compounded monthly
  \item Compounded continuously
  \item Compounded quarterly
  \item Compounded annually
  \end{enumerate}
\end{multicols}
\vspace{0.5in}
\item After the birth of your first child, you are looking to start a college fund with \$10,000 you have saved. There are several banking options, and you want to start an account that will yield the most money in 18 years. Which option is best?
  \begin{multicols}{2}
    \begin{enumerate}
    \item 5\% annually, compounded annually
    \item 4\% annually, compounded continuously
    \item 4.5\% annually, compounded monthly
    \item 4\% annually, compounded quarterly
    \end{enumerate}
  \end{multicols}
  \vspace{3in}
\item Your elderly grandfather reveals to you that he and five friends joined a tontine fifty-five years ago. This tontine carried 6\% interest (compounded continuously) and is currently worth \$1.5 million. How much did each participant pay to join?
\end{enumerate}

\end{document}
