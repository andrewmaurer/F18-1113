\documentclass{article}

\usepackage{enumerate}
\usepackage{amsmath,amsthm,amssymb}
\usepackage{tikz}
\usepackage{pgfplots}
\usepackage{multicol}
% \pgfplotsset{compat=newest}

\usepackage{draftwatermark}
\SetWatermarkScale{4}
\SetWatermarkText{Phantom Quiz}
\usepackage[margin=0.5in]{geometry}

\begin{document}

\noindent \textbf{Name:}\underline{\hspace{2in}} \hfill \textbf{Phantom Quiz November 13}
\\ \\
This is not a real quiz, and it will not be graded. If I gave a quiz on inverse trig functions, it would look like this. (Well, this is a bit longer than what I would give you as a quiz.)
\begin{enumerate}
\item (4 points) Answer the following True or False questions. Write out the entire word; if I can't read it, I can't grade it.
  \begin{enumerate}
    \setlength\itemsep{2em}
  \item \underline{\hspace{1in}} The function $\arcsin(x)$ is the inverse function to $\sin(x)$.
  \item \underline{\hspace{1in}} It is true that $\arctan(x) = \arcsin(x) / \arccos(x)$.
  \item \underline{\hspace{1in}} It is true that $\arcsin^2(x) + \arccos^2(x) = 1$
  \item \underline{\hspace{1in}} It is true that $\sin(\arcsin(x)) = x$
  \end{enumerate}
  \vspace{0.2in}
\item (4 points) A right triangle has an angle $\theta$. The opposite side is length 4, and the adjacent side is 7.
  \begin{enumerate}
  \item Draw the triangle, labeling known sides and angles.
    \vspace{1in}
  \item What is the angle $\theta$, measured in radians? 
  \end{enumerate}
  \vspace{1in}
\item Consider the line $y = 12 x - 5$
  \begin{enumerate}
  \item Draw the graph of $y = 12x - 5$, and label the angle this graph makes with the $x$-axis with the letter $\theta$.
    \vspace{1.5in}
  \item Determine $\theta$. (Hint: Draw a vertical line.)
  \end{enumerate}

  \newpage
\item (5 points) Find all the values of $\theta$ that satisfy the following equations:
  \begin{enumerate}
    \setlength\itemsep{15em}
  \item $\cos^2(2\pi \cdot \theta) + 2\cos(2\pi\cdot \theta) + 1 = 0$
  \item $\ln(\sin(\theta)) - \ln(\cos(\theta)) = 4$
    \vspace{15em}
  \end{enumerate}
\item (4 points) Compute the following values, or explain why no such value exists.
  \begin{enumerate}
    \setlength\itemsep{10em}
  \item $\sin(\arcsin(\pi/3))$
  \item $\tan(\arcsin(0.5))$
    \vspace{10em}
  \end{enumerate}
\item If $\arcsin(x) < 0$, what quadrant does $\arcsin(x)$ lie in?
\end{enumerate}

\end{document}
