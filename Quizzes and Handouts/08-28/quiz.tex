\documentclass{article}

\usepackage{enumerate}
\usepackage{amsmath,amsthm,amssymb}
\usepackage{tikz}
\usepackage{pgfplots}
\usepackage{multicol}
%\pgfplotsset{compat=newest}

\usepackage[margin=1in]{geometry}

\begin{document}

\noindent \textbf{Name:}\underline{\hspace{2in}} \hfill \textbf{Quiz 1: August 28}

\begin{enumerate}
\item The following is the graph of $y = - x \cdot (x-1) \cdot (x+1)$.
  \begin{multicols}{2}
  \begin{tikzpicture}
    \begin{axis}[domain=-2:2,axis lines=middle]
      \addplot+[mark=none] {-x * (x-1) * (x+1)};
    \end{axis}
  \end{tikzpicture}
  \begin{enumerate}
  \item What are the $x$-intercepts of this graph?
    \vspace{0.5in}
  \item What are the $y$-intercepts of this graph?
    \vspace{0.5in}
  \item Express, using interval notation, all points where $-x \cdot (x-1) \cdot x+1$ is a \emph{positive} number.
  \end{enumerate}
\end{multicols}
\vspace{0.5in}
\item The temperature, measured in Fahrenheit, is linearly related to the temperature, measured in Celsius.
  \begin{enumerate}
  \item   Use the fact that $0^\circ C = 32^\circ F$ and $100^\circ C = 212^\circ F$ to express degrees Fahrenheit in terms of degrees Celsius. (Hint: what is the input variable? what is the output?)
    \vspace{1in}
  \item Use the information from (a) to express to express degrees Celsius in terms of degrees Fahrenheit.
    \vspace{1in}
  \item Is the temperature, measured in Fahrenheit, ever equal to the temperature, measured in Celsius? If yes, give the value. If no, explain why not.
  \end{enumerate}
  \vspace{1in}
\item Can the graph of a function ever have more than one $x$-intercept? More than one $y$-intercept? Explain.

\end{enumerate}

\end{document}
