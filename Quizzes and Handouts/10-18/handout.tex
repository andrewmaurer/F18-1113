\documentclass{article}

\usepackage{enumerate}
\usepackage{amsmath,amsthm,amssymb}
\usepackage{tikz}
\usepackage{pgfplots}
\usepackage{multicol}
%\pgfplotsset{compat=newest}

\usepackage[margin=0.5in]{geometry}

\begin{document}

\noindent \textbf{Name:}\underline{\hspace{2in}} \hfill \textbf{Handout October 18}
\\ \\

\begin{enumerate}
\item Draw the terminal side of the given angle, where the initial angle is provided.
  \begin{multicols}{2}
    \begin{enumerate}
    \item $\theta = 120^\circ$
      \vspace{0.5in}
    \item $\theta = 2 \pi / 3$ rad
      \vspace{0.5in}
    \item $\theta = 240^\circ$
      \vspace{0.5in}
    \item $\theta = -9\pi/4$ rad
      \vspace{0.5in}
    \end{enumerate}
  \end{multicols}
  \vspace{0.5in}
\item Convert the following radians to degrees.
  \begin{multicols}{3}
    \begin{enumerate}
      \setlength\itemsep{0.8in}
    \item $-4/3 \pi$ rad
    \item $15/2 \pi$ rad
    \item $8/3 \pi$ rad
    \item $15$ rad
    \item $-200$ rad
    \item $20$ rad
    \end{enumerate}
  \end{multicols}
  \vspace{0.5in}
\item Convert the following degrees to radians.
  \begin{multicols}{3}
    \begin{enumerate}
      \setlength\itemsep{0.8in}
    \item $-30^\circ$
    \item $150^\circ$
    \item $500^\circ$
    \item $15\pi^\circ$
    \item $15 / \pi ^\circ$
    \item $e^\circ$
    \end{enumerate}
  \end{multicols}
  \vspace{0.5in}
\item Which of the following angles, when drawn in standard position, are coterminal to $\pi/4$ rad? (Circle them)
\\ \\
\noindent $-3 \pi / 4$ rad \hfill $5 \pi / 4$ rad \hfill $5 \pi / 8$ rad \hfill $-11 \pi / 4$ rad
\\ 
\item Which of the following angles, when drawn in standard position, are coterminal to $20^\circ$? (Circle them)
  \\ \\
\noindent $340^\circ$ \hfill $380^\circ$ \hfill $-20^\circ$ \hfill $-340^\circ$ \hfill $-380^\circ$
\\ 
\item What angle is complementary to the given angle? Supplementary?
  \begin{multicols}{3}
    \begin{enumerate}
    \item $\pi/4$ rad
      \vspace{0.3in}
    \item $\pi/3$ rad
      \vspace{0.3in}
    \item $20^\circ$
      \vspace{0.3in}
    \item $33^\circ$
      \vspace{0.3in}
    \item 1 rad
      \vspace{0.3in}
    \item $10\pi^\circ$
    \end{enumerate}
  \end{multicols}
  \newpage
\item A circle has radius 4. Answer the following questions, and draw a picture
  \begin{multicols}{2}
  \begin{enumerate}
    \setlength\itemsep{1.2in}
  \item What is the circumference of the circle?
  \item What is the arc-length of a segment measuring $50^\circ$?
  \item What is the arc-length of a segment measuring $3\pi/10$ rad?
  \item What is the arc-length of a segment measuring $15\pi/4$ rad?
  \end{enumerate}
\end{multicols}
\vspace{1in}
\item A circle has radius 3.
  \begin{multicols}{2}
    \begin{enumerate}
      \setlength\itemsep{1.2in}
    \item What is the area of the circle?
    \item What is the area of a sector measuring $10^\circ$
    \item What is the area of a sector measuring $2\pi/3$ rad?
    \item What is the area of a sector measuring $18 \pi / 5$ rad?
    \end{enumerate}
  \end{multicols}
  \vspace{1in}
\item You went for a 100m bike ride. Your bike's wheel's radius is 0.4m.
  \begin{enumerate}
    \setlength\itemsep{1in}
  \item What is the circumference of your bike's tire? What are the units?
  \item How many times did your bike's tire rotate on your bike ride?
  \item What quantity of rotation did your bike's wheel go through in radians? In degrees? 
  \end{enumerate}
\end{enumerate}

\end{document}
